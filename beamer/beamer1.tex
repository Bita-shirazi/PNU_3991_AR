\documentclass{beamer}


\usetheme[useblacktitletext]{diepen}

\usepackage{graphicx}

% Copyright (C) 2018-2020 Pasquale Claudio Africa and the LaTeX community.
% A full list of contributors can be found at
%
%     https://github.com/elauksap/focus-beamertheme
% 
% This file is part of beamerthemefocus.
% 
% beamerthemefocus is free software: you can redistribute it and/or modify
% it under the terms of the GNU General Public License as published by
% the Free Software Foundation, either version 3 of the License, or
% (at your option) any later version.
% 
% beamerthemefocus is distributed in the hope that it will be useful,
% but WITHOUT ANY WARRANTY; without even the implied warranty of
% MERCHANTABILITY or FITNESS FOR A PARTICULAR PURPOSE. See the
% GNU General Public License for more details.
% 
% You should have received a copy of the GNU General Public License
% along with beamerthemefocus. If not, see <http://www.gnu.org/licenses/>.

\mode<presentation>


% DEFINE COLORS. ---------------------------------------------------------------
\definecolor{main}{RGB}{64, 64, 64}
\definecolor{background}{RGB}{239, 239, 239}

\definecolor{alert}{RGB}{180, 0, 0}
\definecolor{example}{RGB}{0, 110, 0}


% SET COLORS. ------------------------------------------------------------------
\setbeamercolor{normal text}{fg=main, bg=background}
\setbeamercolor{alerted text}{fg=alert}
\setbeamercolor{example text}{fg=example}

\setbeamercolor{titlelike}{fg=background, bg=main}
\setbeamercolor{frametitle}{parent={titlelike}}

\setbeamercolor{footline}{fg=background, bg=main}

\setbeamercolor{block title}{bg=main!80!background, fg=background}
\setbeamercolor{block body}{bg=main!10!background, fg=main}

\setbeamercolor{block title alerted}{bg=alert, fg=background}
\setbeamercolor{block body alerted}{bg=alert!10!background, fg=main}

\setbeamercolor{block title example}{bg=example, fg=background}
\setbeamercolor{block body example}{bg=example!10!background, fg=main}

\setbeamercolor{itemize item}{fg=main}
\setbeamercolor{itemize subitem}{fg=main}

\setbeamercolor{enumerate item}{fg=main!70!black}
\setbeamercolor{enumerate subitem}{fg=main!70!black}

\setbeamercolor{description item}{fg=main!70!black}
\setbeamercolor{description subitem}{fg=main!70!black}

\setbeamercolor{caption name}{fg=main}

\setbeamercolor{section in toc}{fg=main}
\setbeamercolor{subsection in toc}{fg=main}
\setbeamercolor{section number projected}{bg=main}
\setbeamercolor{subsection number projected}{bg=main}

\setbeamercolor{bibliography item}{fg=main}
\setbeamercolor{bibliography entry author}{fg=main!70!black}
\setbeamercolor{bibliography entry title}{fg=main}
\setbeamercolor{bibliography entry location}{fg=main}
\setbeamercolor{bibliography entry note}{fg=main}

\mode<all>

\title{E-research methods,strategies,and issuse by Terry Anderson, Heather Kanuka}
\author{bita shirazi}


\begin{document}
  {%
    \setbeamertemplate{headline}{}
    \frame{\titlepage}
  }

  \begin{frame}
\title{chapter thirteen}

the participable that then the absen
In contrast, while "not seeing the participants" is seen as an advantage by Van Nuys, Jacobson (1997) contends that it is arguable that there is no such thing as an online focus group, only moderated online discussions "given the absence of that telltale body language." While not wishing to debate nomenclature, Jacobson maintains that Net-based focus groups are not a substitute for face-to-face focus groups. Rather, Net-based focus groups are "simply an additional tool in the box... and under the right circumstances, clients get a substantial bang for their buck (and they get to sleep in their own beds at night, too!)."

An example of a successful Net-based focus group is shared by Roger Rezabek (2000) (see  $ http://qualitative-research.net/fqs/fqs-eng.htm$). Rezabek conducted a focus group as part of a dissertation research project. He cited the following benefits of conducting a focus group over the Net:

	The individuals who participated in the electronic focus group lived in widely dispersed parts of the U.S. from Nebraska to Maine to Florida, and could not have been brought together physically unless a large sum of money had been available for the travel and time necessary. Although it might have been possible to conduct this exercise using video conferencing equipment, the cost of doing that would have also been substantial. The latter techniques would have allowed a focus group to work in a synchronous or live manner, and would have been possible within a limited amount of time-a half day perhaps over videoconferencing. Two work days, however, would have been necessary, including travel time, if everyone would have been brought together into one location. But expenses are a real part of research, and often, the most economical method becomes the best method to employ. The online focus group experience provided a very economical method to conduct this part of the research, and resulted in vital findings that helped focus and clarify the rest of the study. (Rezabek, 2000)


As expected, Rezabek also noted drawbacks, including "lack of timeliness from beginning to the end of the process, sporadic participation and loss of participation at times by certain members of the group, and variable interaction among the participants."

Another drawback to text-based focus groups is the inability to ask as many questions as in a face-to-face focus group. First, due to the amount of response time and additional time required to get up and running," the number of questions that can be asked is often limited to a maximum of ten, with the optimum number being between three and five. When there are more than five questions, the time commitment increases especially for asynchronous text-based focus groups-resulting in a greater attrition rate. More questions will also require better group processing skills to keep the discussion flowing and lively. In the Rezabek example, three questions were asked over the duration of two and a half months. Rezabek made the following observations:


	During that two and a half month period, three questions were addressed by the focus group, in addition to the housekeeping and getting acquainted aspects at the beginning. A lot can happen in more than two months, and one can forget what comments were made in April by the time your discussion gets to June. Although the timeframe could have been compressed, the discussion just didn't progress that rapidly in order to deal with each of the issues and questions in a shorter period of time.
	
	One of the reasons for the length of the discussions had to do with the sporadic participation that resulted when one member or another didn't participate for a week or two. Although that was allowed and understood, the loss of participants tended to slow the discussion down. Participants temporarily left the discussion because of such circumstances as attending conferences, final exam preparation and grading, semester break, etc. There is little time during the year when such activities will not affect individuals engaged in higher education. But the result was to prolong the online discussions somewhat. And, since the focus group was comprised of specific individuals, their input and reflection on the topic was important.


Finally, according to Fowler (1995), the data from focus groups are often diffuse and hard to work with. This problem becomes even greater with Net-based focus group data and in particular, text-based synchronous discussion data. Both text-based asynchronous and synchronous focus groups tend to suffer from fragmented discussions making data analysis difficult. This results in differences in typing speed among participants and confusion about taking turns. The key to avoiding, or at least reducing, fragmented threads is to have a clear set of objectives and a moderator who can keep the conversation focused, fluid, and on topic.

As with any new technique, experimentation to determine the effectiveness of Net-based focus groups involves an initial trial-and-error process. The following section is a discussion of what we currently know about the process and the resources necessary to carry out a successful Net-based focus group.


  \end{frame}
  
    \begin{frame}
    
If e-researchers do not have prior experience in conducting focus groups, it will be very useful to introduce themselves to the process by observing both face-to-face and Netbased focus groups. Face-to-face focus groups can be difficult to facilitate in a way that results in useful data, but obtaining useful data in Net-based focus groups is even more difficult! If the e-researcher does not facilitate the discussion properly, the data will not be useful. When this happens, a number of problems can arise. First, most participants will be unwilling to volunteer a second time, resulting in an inability to collect data, which can delay-or even bring an end to-the research project. Second, when other e-researchers invite prospective focus group members, participants may refuse to participate based on their (or others") prior negative experiences. Finally, neither researchers nor participants have time to waste and doing so is inconsiderate-if not unethical. For these reasons, it is important that the e-researcher acquire the necessary knowledge and skills to facilitate a successful Net-based focus group.


  \end{frame}
  
    \begin{frame}
  
In a face-to-face focus group, the researcher will usually invite six to twelve participants. We have found that in an asynchronous text-based focus group the size should
be reduced slightly (e.g., six to eight). The reason is that in a Net-based textual environment, reading the combined messages from twelve active participants can be onerous. The time commitment may become greater than expected, resulting in feelings of resentment or in withdrawal by some group members. Alternatively, the focus group should not be too small. Having less than four members makes it difficult to garner the critical mass of input and discussion necessary to build and hone collective ideas and reactions. We have also found that asynchronous Net-based focus groups tend to have a lower participation rate and a higher attrition rate than face-to-face focus groups. Given these factors, the e-researcher should consider inviting twelve to fifteen participants, anticipate that eight to twelve will agree to participate, and assume that three to five will drop out during the study. Of course, whether or not this occurs will depend on the characteristics of and commitment from the targeted group. The ideal focus group would be one in which the e-researcher has a good understanding of the characteristics (.e., motivation, time, likelihood of commitment) of the targeted participants and is able to accurately predict participation. Unfortunately, this is not often the case, and the prudent e-researcher tends to over-sample the target population to insure adequate participation in focus groups.


  \end{frame}
  
  \begin{frame}
As mentioned, online text-based focus group interviews can be conducted either asynchronously or synchronously. While there are advantages and disadvantages to both, asynchronous text-based focus group interviews tend to be more successful than synchronous text-based focus group interviews. 

  \begin{frame}
  
Reasons for this include:
  \end{frame}

\begin{itemize}
\item Asynchronous group participants can take the time to reflect on postings, which can result in more thoughtful responses.
\item Asynchronous focus group participants can respond to questions when it is most convenient, through a capacity to time shift.
	The ideal number of participants for asynchronous focus groups is larger, resulting in greater ability to determine shared views 
\item There is a need for fast and accurate typing skills in synchronous communication (minimum of twenty words per minute is recommended), whereas this is not as necessary for asynchronous communication.
\item Synchronous focus group interview topics need to be kept simple, because participants need to respond quickly. This ensures that the discussion thread is not
	lost. 
\item There is a need for more active and responsive moderating in a synchronous
	focus group, since the facilitator must think and act on the fly.
\end{itemize}

  \end{frame}
  
\begin{frame}

The main weakness of synchronous text-based focus group interviews, however, is that they can easily degenerate into noncontiguous discussions. Noncontiguous dis
  \end{frame} 
\end{document}